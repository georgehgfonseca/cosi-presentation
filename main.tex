\documentclass[mathsans,times]{beamer}

\usepackage[utf8]{inputenc}
\usepackage{beamerthemesplit}
\usepackage[portuges]{babel}
\usepackage{tikz}
\usepackage{color}
\usepackage{xcolor}
% Para alterar o idioma das palavras reservadas dos algoritmos para inglês troque portuguese por english 
\usepackage[portuguese,onelanguage,ruled,vlined,linesnumbered]{algorithm2e}
\usepackage{float}
\usepackage{xcolor,listings}

\usepackage[alf]{abntex2cite}		% Citações padrão ABNT
\usepackage{pifont}%
\usepackage[fleqn]{mathtools}
\usepackage{amsmath}
\usepackage{amssymb}
\usepackage{amsfonts}
\usepackage{mathrsfs}
%\usepackage{subfig}
%\usepackage{bm}

%######################################################
%DEfinição de comandos
\newcommand{\cmark}{\ding{51}}%
\newcommand{\xmark}{\ding{55}}%

\DeclarePairedDelimiter\ceil{\lceil}{\rceil}
\DeclarePairedDelimiter\floor{\lfloor}{\rfloor}
%######################################################

\lstdefinestyle{python}{
	language=Python,
	showspaces=false,
	basicstyle=\footnotesize,
	stringstyle=\color{stringgreen},
	showstringspaces=false,
	numbers=left,
	captionpos=t,
	xleftmargin=0.3cm,
	frame = single, 
	backgroundcolor=\color{gray!5},
	keywordstyle=\color{blue},
	numberstyle=\tiny,
	caption={main.py},
	title=main.py,
	commentstyle=\color{gray}}
\lstset{
	style=python
}

\usepackage{pgfplots}
\usepackage{subfigure}
\usepackage{float}
\usepackage{color}

\usepackage{verbatim}
\usetheme{CambridgeUS}
\usepackage{cancel}
\usepackage{lmodern}

\hyphenpenalty 100000

\setbeamertemplate{headline}
{
	\leavevmode%
	\hbox{%
		\begin{beamercolorbox}[wd=1.0\paperwidth,ht=2.65ex,dp=1.5ex,right]{section in head/foot}%
			\usebeamerfont{section in head/foot}\insertsectionhead\hspace*{2ex}
		\end{beamercolorbox}%
	}%
	\vskip0pt%
}

\definecolor{commentgreen}{HTML}{008000}

% A cor da UFOP foi definida como a cor do tema
\definecolor{themecolor}{RGB}{171,0,40}
\setbeamercolor{title}{bg=themecolor,fg=white}
\setbeamercolor{frametitle}{fg=themecolor}
\setbeamercolor{title in head/foot}{fg=themecolor}
\setbeamercolor{section in head/foot}{fg=white, bg=themecolor}
\setbeamercolor{block title}{parent=normal text,bg=themecolor,fg= white}
\setbeamercolor{item}{fg=themecolor}
\setbeamercolor{subitem}{fg=themecolor}

\let\oldfootnote\footnote
\renewcommand\footnote[1][]{\oldfootnote[frame,#1]}
\setbeamertemplate{footline}[frame number]{}

\setbeamertemplate{navigation symbols}{} %remove navigation symbols

\makeatletter
\newcommand{\subalign}[1]{%
	\vcenter{%
		\Let@ \restore@math@cr \default@tag
		\baselineskip\fontdimen10 \scriptfont\tw@
		\advance\baselineskip\fontdimen12 \scriptfont\tw@
		\lineskip\thr@@\fontdimen8 \scriptfont\thr@@
		\lineskiplimit\lineskip
		\ialign{\hfil$\m@th\scriptstyle##$&$\m@th\scriptstyle{}##$\crcr
			#1\crcr
		}%
	}
}

%##########################################
% SLIDES DE INÍCIO DE SEÇÃO
\newtranslation[to=portuges]{Section}{Seção}
\newtranslation[to=portuges]{Subection}{Subseção}

\AtBeginSection
{
	\setbeamertemplate{section page}[mine]
	\frame{\sectionpage}
	
}
\AtBeginSubsection
{
	\setbeamertemplate{subsection page}[mine]
	\frame{\subsectionpage}
	
}


\defbeamertemplate{section page}{mine}[1][]{
	\begin{centering}
		%{\usebeamerfont{section name}\usebeamercolor[fg]{normal text}\sectionname~\insertsectionnumber}
		\vskip1em\par
		\begin{beamercolorbox}[shadow=true,sep=12pt,center]{block title}
			\insertsectionnumber.~\insertsection\par
		\end{beamercolorbox}
	\end{centering}
}


\defbeamertemplate{subsection page}{mine}[1][]{
	\begin{centering}
		%{\usebeamerfont{section name}\usebeamercolor[fg]{normal text} \insertsection}
		\vskip1em\par
		\begin{beamercolorbox}[shadow=true,sep=12pt,center]{block title}
			\insertsectionnumber.\insertsubsectionnumber.~\insertsubsection\par
		\end{beamercolorbox}
	\end{centering}
}
%##########################################

\title{\textbf{Template para Apresentações}}
\author{Autor(es)}
\institute{Nome da matéria \\ Nome do curso/departamento \\ Universidade Federal de Ouro Preto \\}
\date{Mês de Ano}


\begin{document}
%	\pgfdeclarelayer{background}
%	\pgfsetlayers{background,main}
	

	
	\frame{
		\titlepage
		\vspace{-20mm}
		\begin{flushleft}
			\includegraphics[scale=0.125]{img/logo_ufop.png}
		\end{flushleft}
	}

\section{Texto em Tópicos}

\begin{frame}
	\frametitle{Texto em Tópicos}
	\begin{itemize}
	\item Esse slide apresenta o texto organizado em tópicos e sub-tópicos.
	\vspace{3mm}
	\item O restante do modelo apresenta os seguintes exemplos:
	\begin{itemize} 
		\item Figuras.
		\item Tabelas.
		\item Algoritmos.
		\item Código Fonte.
		\item Modelos Matemáticos.
		\item Referências.
	\end{itemize} 
\end{itemize}
\end{frame}

\begin{frame}
	\frametitle{Texto em Tópicos - \textit{overlays}}
\begin{itemize}
	\item É possível quebrar um slide em vários usando \textit{overlays}:
	\begin{itemize} 
		\item <1-> Figuras.
		\item <2-> Tabelas.
		\item <3-> Algoritmos.
		\item <4-> Código Fonte.
		\item <5-> Modelos Matemáticos.
		\item <6> Referências.
	\end{itemize} 
\end{itemize}
\end{frame}

\section{Figuras}
\begin{frame}
	\frametitle{Figuras}
	\begin{itemize}
		\item Figuras, tabelas e elementos não textuais são trabalhados exatamente como em documentos comuns LaTeX.
		\vspace{3mm}
		\item Pode ser interessante incluir um espaçamento entre o texto e a imagem pelo comando $\backslash$vspace\{3mm\}
		\vspace{3mm}
		\begin{figure}[!htpb]
			\centering\includegraphics[width=0.3\linewidth]{img/donald-duck.png}
			\caption{É possível inserir legendas também.}
		\end{figure}
		
	\end{itemize}
\end{frame}

\begin{frame}
	\frametitle{Duas colunas: figura e texto}
	\footnotesize
	\begin{columns}[onlytextwidth]
		\begin{column}{0.45\textwidth}
			\begin{itemize}
				\item Deste lado fica o texto, do outro a figura:
				\begin{itemize}
					\item item 1
					\item item 2
					\item item 3
					\item item 4
					\item item 5
				\end{itemize}
			\end{itemize}
		\end{column}
		\begin{column}{0.5\textwidth}
			\centering
			\includegraphics[width=0.6\textwidth]{img/sponge-bob.png}
		\end{column}
	\end{columns}
\end{frame}

\section{Tabelas}

\begin{frame}
	\frametitle{Tabelas}
	\begin{itemize}
		\item Uma excelente ferramenta para a criação de tabelas é o \url{https://www.tablesgenerator.com/}
		\vspace{3mm}
		\item Um exemplo de tabela é dado abaixo
		\vspace{5mm}
		\begin{table}[htb]
			\centering
			\caption{É possível inserir legendas também.}
			\label{tab:resultados}
			\begin{tabular}{lrrrr}
				\hline
				Instância       &    Algoritmo A &     Algoritmo B &    Algoritmo C \\\hline
				\textit{Arquivo 1}  &  1.00$\pm$0.7  &    1.90$\pm$0.5 &   \textbf{0.90$\pm$0.3} \\
				\textit{Arquivo 2}  &  6.80$\pm$0.5  &    8.60$\pm$1.4 &   \textbf{5.90$\pm$0.2} \\
				\textit{Arquivo 3}  &  \textbf{7.90$\pm$1.2}  &   13.25$\pm$3.1 &   8.50$\pm$1.7 \\
				\textit{Arquivo 4}  &  8.00$\pm$0.0  &   12.70$\pm$0.2 &  \textbf{6.10$\pm$1.0} \\\hline
			\end{tabular}
		\end{table}
	\end{itemize}
\end{frame}

\section{Algoritmos}

\begin{frame}
	\frametitle{Algoritmos}
	\begin{itemize}
		\item Siga as convenções para pseudocódigos conforme \url{https://link.springer.com/content/pdf/bbm\%3A978-1-4471-5173-9\%2F1.pdf}
		\vspace{3mm}
		\item Para trocar o idioma das palavra-chave altere o idioma de \textit{portuguese} para \textit{english} na linha:\\
		$\backslash$usepackage[portuguese, ...]\{algorithm2e\}
	\end{itemize}
\end{frame}

\begin{frame}
	\frametitle{Algoritmos}
\begin{itemize}
	\item Pseudocódigo do algoritmo de ordenação Bubble-Sort:
\end{itemize}
\hspace{1mm}
\begin{algorithm}[H]
	\footnotesize
	\KwIn{(i) Um vetor $V$.}
	\KwOut{(i) O vetor $V$ ordenado crescentemente.}
	\For{$i = 0$ até $|V| - 1$} {
		$trocou \leftarrow$ \textbf{false};\\
		\For{$j = 0$ até $|V| - 1$}{
			\If{$V[j + 1] > V[j]$}{
				$aux \leftarrow V[j]$;\hspace{3mm}\textcolor{commentgreen}{$\triangleright$ Troca as posições $i$ e $i + 1$ de $V$}\\
				$V[j] \leftarrow V[j + 1]$;\\
				$V[j + 1] \leftarrow aux$;\\
				$trocou \leftarrow$ \textbf{true};\\ 
			}
		}
		\If{$trocou$}{
			\textbf{break};\hspace{3mm}\textcolor{commentgreen}{$\triangleright$ $V$ já está ordenado}\\
		}
	}
	\Return $V$;
	\footnotesize
	\caption{BUBBLE-SORT.}
	\label{alg:multistart}
\end{algorithm}
\end{frame}

\section{Código-Fonte}

\begin{frame}[fragile]
	\frametitle{Código-Fonte}
	\begin{itemize}
	\item Segue um exemplo de implementação do algoritmo Bubble-Sort na linguagem Python
\end{itemize}
\begin{lstlisting}
def bubble_sort(V): #Complexidade O(N^2)
  for i in range(len(V) - 1):
    trocou = False
    for j in range(len(V) - 1):
      if V[j + 1] < V[j]:
        aux = V[j] 
        V[j] = V[j + 1]
        V[j + 1] = aux
        trocou = True
    if not trocou:
      break
  return V
	\end{lstlisting}
\end{frame}

\section{Modelos Matemáticos}

\begin{frame}
	\frametitle{Modelos Matemáticos}
\textbf{Problema da mochila}
\vspace{3MM}
\begin{itemize}
	\item Conjuntos e parâmetros
	\begin{table}[H]
		\begin{tabular}{ll}
			$\mathcal{I}$   & Itens \\ 
			$w_{i}$         & Peso do item $i$\\
			$b_{i}$         & Benefício do item $i$\\
			$C$             & Capacidade da mochila\\
		\end{tabular} 
	\end{table}
	\item Variáveis
	%Outra opção para apresentar variáveis binárias
	\begin{equation*}
		x_i=\begin{cases}
			1 & \parbox[t]{.6\textwidth}{\mbox{se o item $i \in \mathcal{I}$ está na mochila.}}\\
			0 & \mbox{caso contrário.}
		\end{cases}
	\end{equation*}
	
	% 	\begin{table}[H]
		% 	\centering
		% 	\begin{tabular}{ll}
			% 		$x_i$   & 1 se o item $i \in \mathcal{I}$ está na mochila; 0 caso contrário \\ 
			% 	\end{tabular} 
		% 	\end{table}
\end{itemize}
\end{frame}

\begin{frame}
	\frametitle{Modelos Matemáticos}
\begin{itemize}
	\item Restrições
	\begin{enumerate}
		\item Capacidade da mochila:
		\begin{equation}
			\begin{split} 
				\label{eq:r1}
				\sum_{i \in \mathcal{I}} w_i \times x_i \leq C
			\end{split}
		\end{equation}
	\end{enumerate}
	
	\item Função objetivo
	\begin{equation}
		\begin{split} 
			\label{eq:obj}
			\textbf{max} \sum_{i \in \mathcal{I}} b_i \times x_i
		\end{split}
	\end{equation}
\end{itemize}
\end{frame}

\section{Referências}

\begin{frame}
	\frametitle{Referências}
\begin{itemize}
	\item Sempre que se utilizar do trabalho de outrem é necessário incluir uma citação, que pode fazer parte do texto ou não.
	\begin{description}
		\item [Textual] \citeonline{fonseca2017} apresenta uma nova formulação para o problema. 
		\item [Não-textual] O problema de programação de horários é classificado como NP-Difícil \cite{even1975}.
	\end{description}
\end{itemize}
\end{frame}

\begin{frame}[allowframebreaks]{Referências}
	\bibliography{bibliografy}
\end{frame}

\end{document}