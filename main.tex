\documentclass[mathsans,times]{beamer}

\input epsf

\usepackage[utf8]{inputenc}
\usepackage{beamerthemesplit}
\usepackage{optprog}
\usepackage[portuges]{babel}
\usepackage{tikz}
\usepackage{color}
% Para alterar o idioma das palavras reservadas dos algoritmos para inglês troque portuguese por english 
\usepackage[portuguese,onelanguage,ruled,vlined,linesnumbered]{algorithm2e}
\usepackage{float}
\usepackage{xcolor,listings}

\usepackage[alf]{abntex2cite}		% Citações padrão ABNT
\usepackage{pifont}%
\usepackage[fleqn]{mathtools}
\usepackage{amsmath}
\usepackage{amssymb}
\usepackage{amsfonts}
\usepackage{mathrsfs}

%######################################################
%DEfinição de comandos
\newcommand{\cmark}{\ding{51}}%
\newcommand{\xmark}{\ding{55}}%

\DeclarePairedDelimiter\ceil{\lceil}{\rceil}
\DeclarePairedDelimiter\floor{\lfloor}{\rfloor}
%######################################################

%DEFINIÇÃO DE NÓS NO TIKz
\usetikzlibrary{chains,fit,shapes,calc}
\usetikzlibrary{positioning} 
\usetikzlibrary{arrows}
\tikzset{
  treenode/.style = {align=center, inner sep=0pt, text centered,
    font=\sffamily},
  arn_n/.style = {treenode, circle, white, font=\sffamily\bfseries, draw=black,
    fill=black, text width=1.5em},% arbre rouge noir, noeud noir
	arn_g/.style = {treenode, circle, white, font=\sffamily\bfseries, draw=gray,
    fill=gray, text width=1.5em}% arbre rouge noir, noeud noir
}
%######################################################

\usetheme{lasos}

\title{Template LaTeX para \\Apresentações de TCC}
\author{and }
\author[shortname]{Aluno  \inst{1} \and Orientador \inst{2}}
\institute[shortinst]
{
\inst{1} Curso / UFOP
%affiliation for author1
\and %
\inst{2} Departamento / UFOP
%affiliation for author2
}
\date{Junho de 2020}


\begin{document}

%##############################################
% CONFIGURAÇÕES DO TEMPLATE

% LOGO DO LASOS A ESQUERDA NO SLIDE INICIAL - ok
% BARRA HEADLINE VERMELHO - ok
%BARRA DO TITULO CINZA CLARO - ok

% EXEMPLOS: TOPICOS E SUBTÓPICOS, FIGURAS, TABELAS, ALGORITMOS, CODIGO FONTE, REFERÊNCIAS, AGRADECIMENTOS

%##########################################
% SLIDES DE INÍCIO DE SEÇÃO
\newtranslation[to=portuges]{Section}{Seção}
\newtranslation[to=portuges]{Subection}{Subseção}

\AtBeginSection
{
\setbeamertemplate{section page}[mine]
\frame{\sectionpage}

}
\AtBeginSubsection
{
\setbeamertemplate{subsection page}[mine]
\frame{\subsectionpage}

}

%##########################################

\makenoheadtitle

\section{Texto em Tópicos}
%\subsection{Subtópico}
\frame{
	\frametitle{Texto em Tópicos}
	\begin{itemize}
		\item Esse slide apresenta o texto organizado em tópicos e subtópicos.
		\item O restante do modelo apresenta os seguintes exemplos:
	    \begin{itemize} 
		    \item Figuras.
		    \item Tabelas.
		    \item Algoritmos.
		    \item Código Fonte.
		    \item Modelos Matemáticos.
		    \item Agradecimentos.
		    \item Referências.
		    \vspace{10mm}
    	\end{itemize} 
	\end{itemize}
}

%\subsection{Subtópico}
\frame{
	\frametitle{Texto em Tópicos - overlays}
	\begin{itemize}
		\item Esse slide apresenta o texto organizado em tópicos e subtópicos.
		\item O restante do modelo apresenta os seguintes exemplos:
	    \begin{itemize} 
		    \item <1-> Figuras.
		    \item <2-> Tabelas.
		    \item <3-> Algoritmos.
		    \item <4> Código Fonte.
		    \item <4-6> Modelos Matemáticos.
		    \item <7-> Agradecimentos.
		    \item <7> Referências.
		    \vspace{10mm}
    	\end{itemize} 
	\end{itemize}
}

\section{Figuras}

\frame{
	\frametitle{Figuras}
	\begin{itemize}
		\item Figuras, tabelas e elementos não textuais são trabalhados exatamente como em documentos comuns LaTeX.
		\item Pode ser interessante incluir um espaçamento entre o texto e a imagem pelo comando $\backslash$vspace\{3mm\}
		\vspace{3mm}
		\begin{figure}[!htpb]
	    	\centering\includegraphics[width=0.3\linewidth]{img/example.png}
	    	\caption{É possível inserir legendas também.}
    	\end{figure}

	\end{itemize}
}
\frame{
\frametitle{Duas colunas: Figura e texto}
\footnotesize
\begin{columns}[onlytextwidth]
    \begin{column}{0.45\textwidth}
Deste lado fica o texto, do outro a(s) figura(s)
\begin{itemize}
\item item 1
\item item 2
\end{itemize}
Observe ue neste slide, as figuras estão utilizando \textit{overlays}
  \end{column}
    \begin{column}{0.5\textwidth}
	\includegraphics<2>[width=6cm]{img/figure_1a.png}
	\includegraphics<3>[width=6cm]{img/figure_1b.png}
	\includegraphics<4>[width=6cm]{img/figure_1c.png}
    \end{column}
 \end{columns}

}


\section{Tabelas}

\frame{
	\frametitle{Tabelas}
	\begin{itemize}
		\item Uma excelente ferramenta para a criação de tabelas é o \url{https://www.tablesgenerator.com/}.
	    \item Um exemplo de tabela é dado abaixo.
		\vspace{5mm}
        \begin{table}[htb]
        \centering
        \caption{É possível inserir legendas também.}
        \label{tab:resultados}
        \begin{tabular}{lrrrr}
    	    \hline
            Instância       &    Algoritmo A &     Algoritmo B &    Algoritmo C \\\hline
            \textit{Arquivo 1}  &  1.00$\pm$0.7  &    1.90$\pm$0.5 &   \textbf{0.90$\pm$0.3} \\
            \textit{Arquivo 2}  &  6.80$\pm$0.5  &    8.60$\pm$1.4 &   \textbf{5.90$\pm$0.2} \\
            \textit{Arquivo 3}  &  \textbf{7.90$\pm$1.2}  &   13.25$\pm$3.1 &   8.50$\pm$1.7 \\
            \textit{Arquivo 4}  &  8.00$\pm$0.0  &   12.70$\pm$0.2 &  \textbf{6.10$\pm$1.0} \\\hline
        \end{tabular}
        \end{table}
    \end{itemize}
}

\section{Algoritmos}

\frame{
	\frametitle{Algoritmos}
	\begin{itemize}
	    \item Siga as convenções para pseudocódigos conforme \url{https://link.springer.com/content/pdf/bbm\%3A978-1-4471-5173-9\%2F1.pdf}
	    \item Para trocar o idioma das palavra-chave altere o idioma de \textit{portuguese} para \textit{english} na linha:\\
	    $\backslash$usepackage[portuguese, ...]\{algorithm2e\}
    \end{itemize}
}

\frame{
	\frametitle{Algoritmos}
	\begin{itemize}
	    \item Pseudocódigo do algoritmo de ordenação Bubble-Sort:
		\vspace{2mm}
\end{itemize}

\begin{algorithm}[H]
\footnotesize
	\KwIn{(i) Um vetor $V$.}
	\KwOut{(i) O vetor $V$ ordenado crescentemente.}
	\For{$i = 0$ até $|V| - 1$} {
    	$trocou \leftarrow$ \textbf{false};\\
	    \For{$j = 0$ até $|V| - 1$}{
	        \If{$V[j + 1] > V[j]$}{
	            $aux \leftarrow V[j]$;\hspace{17mm}$\triangleright$ Troca as posições $i$ e $i + 1$ de $V$.\\
	            $V[j] \leftarrow V[j + 1]$;\\
	            $V[j + 1] \leftarrow aux$;\\
	            $trocou \leftarrow$ \textbf{true};\\ 
	        }
	    }
	    \If{$trocou$}{
	        \textbf{break};\hspace{49mm}$\triangleright$ $V$ já está ordenado.\\
	    }
	}
	\Return $V$;
\footnotesize
	\caption{BUBBLE-SORT.}
	\label{alg:multistart}
\end{algorithm}

}

\section{Código-Fonte}

\begin{frame}[fragile]
	\frametitle{Código-Fonte}
	\begin{itemize}
    	\item Segue um exemplo de implementação do algoritmo Bubble-Sort na linguagem Python
	\end{itemize}
\begin{tikzpicture}
\node[draw, ufopred, thick, fill=gray!10, text=black, text width=10.5cm, align=left]{
	\begin{lstlisting}[
	language=Python,
	showspaces=false,
	basicstyle=\small,
	showstringspaces=false
	%numbers=left,
	%numberstyle=\tiny,
	commentstyle=\color{gray}]
def bubblesort(V): #Complexidade O(N^2)
    for i in range(len(V) - 1):
        trocou = False
        for j in range(len(V) - 1):
            if V[j + 1] < V[j]:
                aux = V[j] 
                V[j] = V[j + 1]
                V[j + 1] = aux
                trocou = True
        if not trocou:
            break
    return V
\end{lstlisting}};
\end{tikzpicture}
\end{frame}

\section{Modelos Matemáticos}
\subsection{Modelo descrito no texto}
\frame{
	\frametitle{Modelos Matemáticos}
	PROBLEMA DA MOCHILA
\vspace{3MM}
\begin{itemize}
\item Conjuntos e parâmetros
	\begin{table}[H]
	\begin{tabular}{ll}
		$\mathcal{I}$   & Itens \\ 
		$w_{i}$         & Peso do item $i$\\
		$b_{i}$         & Benefício do item $i$\\
		$C$             & Capacidade da mochila\\
	\end{tabular} 
	\end{table}
\item Variáveis
%Outra opção para apresentar variáveis binárias
\begin{equation*}
x_i=\begin{cases}
1 & \parbox[t]{.6\textwidth}{\mbox{se o item $i \in \mathcal{I}$ está na mochila.}}\\
0 & \mbox{caso contrário.}
\end{cases}
\end{equation*}

% 	\begin{table}[H]
% 	\centering
% 	\begin{tabular}{ll}
% 		$x_i$   & 1 se o item $i \in \mathcal{I}$ está na mochila; 0 caso contrário \\ 
% 	\end{tabular} 
% 	\end{table}
\end{itemize}
}

\frame{
	\frametitle{Modelos Matemáticos}
\begin{itemize}
\item Restrições
\begin{enumerate}
    \item Capacidade da mochila:
	\begin{equation}
	\begin{split} 
	\label{eq:r1}
    \sum_{i \in \mathcal{I}} w_i \times x_i \leq C
	\end{split}
	\end{equation}
\end{enumerate}
	
\item Função objetivo
	\begin{equation}
	\begin{split} 
	\label{eq:obj}
	\textbf{max} \sum_{i \in \mathcal{I}} b_i \times x_i
	\end{split}
	\end{equation}
\end{itemize}
}
\subsection{usando o Optprog}
\frame
{
\frametitle{Localização de Instalações}
\scriptsize 
  
\textbf{Formulação:}\\
\begin{optprog}	
& \objective {\min \sum_{s \in S} \left\{C^f_{s}Y_{s} + \sum_{c \in C} V_{c}(G_{sc}X_{sc}) \right\} +\sum_{c \in C}V_cG_{c}X_{ic}}   
\label{objloc}\\
\vspace{0.2cm}
& \sum_{s \in S} X_{sc} + X_{ic}  & = & 1 , & \forall c \in C \label{locr1}\\
& \sum_{c \in C} X_{sc} & \leq & Y_{s}|C| , & \forall s \in S,  \label{locr1b}\\
& X_{sc} & \in & \{0,1\}, & \forall s \in S, \forall c \in C \label{locr4}\\
& Y_{s} & \in & \{0,1\}, & \forall s\in S \label{locr5}
\end{optprog}

}
\section{Usando o TiKz}
\subsection{Com Overlay}
\frame{
\frametitle{Árvore de BB - overlays}
\centering
\begin{tikzpicture}[->,>=stealth'] 
			\node (P)[arn_n] at (1.5,1.5) {P};
			
			\onslide<2->{\node (S1) [arn_n] at (3.0,0.0) {$S_1$};
				     \draw [->](P) to (S1) ;}
			\onslide<3->{\node (S2) [arn_n] at (0.0,0.0) {$S_2$};
			 	      \draw [->](P) to (S2) ;}
			\onslide<5->{ \draw (P) -- (S1) node [midway, above, sloped] (TextNode) {\scriptsize $x \geq \ceil*{b}$};}
			\onslide<6->{\draw (P) -- (S2) node [midway, above, sloped] (TextNode1) {\scriptsize $x \leq \floor*{b}$};}
			\onslide<8->{\node (LsP) at (1.5,1.9){\tiny \textbf{LS}};
				     \node (LiP) at (1.5,1.1){\tiny \textbf{LI}};
				     \node (LsS1) at (3.0,0.4){\tiny \textbf{LS}};
				     \node (LiS1) at (3.0,-0.4){\tiny \textbf{LI}};
				     \node (LsS2) at (0.0,0.4){\tiny \textbf{LS}};
				     \node (LiS2) at (0.0,-0.4){\tiny \textbf{LI}};}
		

            				
\end{tikzpicture}

}
\subsection{Estático}
\frame{
\frametitle{Problema do fluxo de custo mínimo}

\tikzstyle{vertex}=[circle,fill=black!25,minimum size=20pt,inner sep=0pt]

\tikzstyle{edge} = [draw,thick,->]
\tikzstyle{weight} = [font=\tiny]


\begin{figure}
\label{fig1}
\begin{tikzpicture}[scale=1.8, auto,swap]

    \node[vertex] (e1) at (-3,1){$i$};
    \node[vertex] (e2) at (-2,1){$j$};
    \path[edge] (e1) -- node[weight,midway, below, sloped] {$(u_{ij};c_{ij})$} (e2); 
    \foreach \pos/\name in {{(-3,0)/f}, {(-1,1)/a}, {(1,1)/c},
                            {(-1,-1)/b}, {(1,-1)/d}, {(3,0)/t}}
        \node[vertex] (\name) at \pos {$\name$};
    % Connect vertices with edges and draw weights
    \foreach \source/ \dest /\weight in {f/a/(7;3), f/b/(8;4),b/a/(5;2),a/b/(9;1),
                                         a/c/(7;6), b/c/(5;3),b/d/(6;5),
                                         c/d/(3;7),c/t/(8;3),
                                         d/t/(7;2)}
        \path[edge] (\source) -- node[weight,midway, below, sloped] {$\weight$} (\dest);
\end{tikzpicture}
\caption{Exemplo de rede}
\end{figure}


}

\section{Referências}

\frame{
	\frametitle{Referências}
	\begin{itemize}
		\item Sempre que se utilizar do trabalho de outrem é necessário incluir uma citação, que pode fazer parte do texto ou não.
		\begin{description}
		    \item [Textual] \citeonline{fonseca2017} apresenta uma nova formulação para o problema. 
		    \item [Não-textual] O problema de programação de horários é classificado como NP-Difícil \cite{even1975}.
		\end{description}
%		\item Ao final da apresentação incluir o slide de agradecimentos às instituições que apoiam o LASOS.
    \end{itemize}
}

%\frame{
%	\frametitle{Agradecimentos}
%\begin{figure}
%\centering
%\begin{minipage}{.33\textwidth}
%  \centering
%  \includegraphics[width=.8\linewidth]{img/cnpq}
%%  \caption{CNPq}
%%  \label{fig:test1}
%\end{minipage}%
%\begin{minipage}{.33\textwidth}
%  \centering
%  \includegraphics[width=.4\linewidth]{img/ufop}
%%  \caption{UFOP}
%%  \label{fig:test1}
%\end{minipage}%
%\begin{minipage}{.33\textwidth}
%  \centering
%  \includegraphics[width=.7\linewidth]{img/fapemig}
%%  \caption{FAPEMIG}
%%  \label{fig:test2}
%\end{minipage}
%\end{figure}
%}

\begin{frame}[allowframebreaks]{Referências}

\bibliography{bibliografy}
\end{frame}


\end{document}